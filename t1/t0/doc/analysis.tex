\section{Theoretical Analysis}
\label{sec:analysis}

In this section, the circuit shown in Figure~\ref{fig:rc} is analysed
theoretically. The circuit consists of seven resistors, two voltage sources and two current sources. We will begin by analyzing the circuit by applying the Nodal method and, after that, we will analyze it again using the Mesh method . In order to this we will use Kirchhoff's circuit laws (KVL and KCL) together with Ohm's Law to obtain the theoretical results.
 
\subsection{Nodal Method}
It was assigned potencial 0 to the central node in order to proceed with the analysis. Because of that we can write:
\begin{equation}
  V_b=V_2 - 0 = V_2
\end{equation}
\begin{equation}
  I_b= K_b \cdot V_b= K_b \cdot V_2
\end{equation}

We first start by calculating the values of the conductances of the various resistors:
\begin{equation}
  G_i=1/R_i
\end{equation}

Equation obtain by applying KCL to:
Node number 1
\begin{equation}
  (V_2 - V_1)\cdot G_1 -I_a =0
  \label{eq:kcl}
\end{equation}

Node number 2
\begin{equation}
  -(V_2 -V_1)\cdot G_1 -V_b \cdot G_3 + (V_3-V_2)\cdot G_2 =0 
  \Leftrightarrow -(V_2 -V_1)\cdot G_1 -V_2 \cdot G_3 + (V_3-V_2)\cdot G_2 =0
  \label{eq:kcl2}
\end{equation}

Node number 3
\begin{equation}
  I_b - (V_3-V_2) \cdot G_2=0 \Leftrightarrow V_2(G_2 + K_b) - G_2 \cdot V_3
  \label{eq:kcl3}
\end{equation}

Node number 4
\begin{equation}
  -K_b\cdot V_2 - V_4 \cdot G_5 = - I_d
  \label{eq:kcl4}
\end{equation}

Node number 5
\begin{equation}
  (V_6 \cdot V_5) \cdot G_7 + I_vc - I_d=0
  \label{eq:kcl5}
\end{equation}

Node number 6
\begin{equation}
 	V_5 \cdot G_7 + V_6\cdot (-G_6 -G_7) + V_7 \cdot G_6 =0
  \label{eq:kcl6}
\end{equation}

Node number 7
\begin{equation}
  V_1 \cdot G_1 - G_1 \cdot V_2 - G_6 \cdot V_6 + V_7 \cdot (G_4 + G_6)=0
  \label{eq:kcl7}
\end{equation}

Matricial Equation obtained using the nodal method:
\begin{gather}
	\begin{bmatrix}
		G_1 & -G_1 & 0 & 0 & 0 & -G_6 & G_4 + G_6 \\
		G_1 & -G_1 - G_2 - G_3 & G_2 & 0 & 0 & 0 & 0 \\
		0 & G_2 + K_b & -G_2 & 0 & 0 & 0 & 0 \\
		0 & K_b & 0 & G_5 & 0 & 0 & 0 \\
		0 & 0 & 0 & 0 & G_7 & -G_6 - G_7 & G_6 \\
		1 & 0 & 0 & 0 & 0 & 0 & -1 \\
		0 & 0 & 0 & 0 & 1 & -G_6 & K_c \\
	\end{bmatrix}
	\begin {bmatrix} V_1 \\ V_2 \\ V_3 \\ V_4  \\ V_5 \\ V_6 \\ V_7 \end{bmatrix}
	=
	\begin {bmatrix} 0  \\ 0  \\ 0  \\ I_d \\ 0  \\ V_a \\ 0 \end{bmatrix}
\end{gather}

\begin{table}[h]
  \centering
  \begin{tabular}{|l|r|}
    \hline    
    {\bf Node} & {\bf Voltage[V]} \\ \hline
    \input{../mat/Nodal_tab}
  \end{tabular}
  \caption{Nodal Method}
  \label{tab:nodal}
\end{table}

		
\subsection{Mesh Method}

Equation obtain by applying KVL to:
Mesh A
\begin{equation}
  -V_a+ R_1\cdot I_A + R_3 \cdot (I_A+I_B)+ R_4 \cdot (I_A+I_C) =0
  \label{eq:kvl}
\end{equation}

Mesh B
\begin{equation}
  I_B = I_b = K_b \cdot V_b = Kb \cdot R_3 \cdot (I_A + I_B)
  \label{eq:kvl2}
\end{equation}

Mesh C
\begin{equation}
  R_4 \cdot (I_C + I_A) + R_6 \cdot I_C + R_7 \cdot I_C - K_c \cdot I_C = 0
  \label{eq:kvl3}
\end{equation}

Mesh D
\begin{equation}
  I_D =I_d
  \label{eq:kvl4}
\end{equation}

Matricial Equation obtained using the mesh method:
\begin{gather}
	\begin{bmatrix}
		R_1 + R_3 + R_4 & R_3 & R_4 & 0  \\
		K_b \cdot R_3 & K_b \cdot R_3 - 1& 0 & 0 \\
		R_4 & 0 & R_4 + R_6 + R_ 7 - K_c & 0 \\
		0 & 0 & 0 & 1 \\
	\end{bmatrix}
	\begin {bmatrix} I_A \\ I_B \\ I_C \\ I_D \end{bmatrix}
	=
	\begin {bmatrix} V_a \\ 0 \\ 0 \\ I_d \end{bmatrix}
\end{gather}

  
\begin{table}[h]
  \centering
  \begin{tabular}{|l|r|}
    \hline    
    {\bf Mesh} & {\bf Current[A]} \\ \hline
    \input{../mat/Mesh_tab}
  \end{tabular}
  \caption{Mesh Method}
  \label{tab:mesh}
\end{table}

